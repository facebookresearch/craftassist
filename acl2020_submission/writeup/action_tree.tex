\clearpage
\onecolumn

\section{Action Tree structure}
\label{sec:action_tree}

%\begin{figure*}[ht!]
%    \centering
%     \includegraphics[natwidth=\linewidth]{figures/ActionGrammar.png}
%    \includegraphics[width=0.70\linewidth]{figures/AS.png}
%    \includegraphics[width=0.30\linewidth]{figures/NOT_AS.png}
%    \caption{Action space grammar}
%    \label{fig:action_grammar}
%\end{figure*}
This section describes the details of logical form of each action.
We support three dialogue types: HUMAN\_GIVE\_COMMAND, GET\_MEMORY and PUT\_MEMORY.
The logical form for actions has been pictorially represented in Figures: \ref{fig:AS} and \ref{fig:NOT_AS}

We support the following actions in our dataset : Build, Copy, Dance, Spawn, Resume, Fill, Destroy, Move, Undo, Stop, Dig and FreeBuild.
A lot of the actions use  ``location'' and ``reference\_object'' as children in their logical forms. To make the logical forms more presentable, we have shown the detailed representation of a ``reference\_object'' (reused in action trees using the variable: ``REF\_OBJECT'') in Figure \ref{fig:ref_obj} and the representation of ``location'' (reused in action trees using the variable: ``LOCATION'') in figure \ref{fig:location}. The representations of actions refer to these variable names in their trees.


\begin{figure}[h]
    \centering
    \includegraphics[width=15cm,height=20cm,keepaspectratio]{figures/ref_obj.png}
    \caption{Logical form of a reference\_object child}
    \label{fig:ref_obj}
\end{figure}


\begin{figure}[h]
    \centering
    \includegraphics[width=15cm,height=20cm,keepaspectratio]{figures/location.png}
    \caption{Logical form of a location child}
    \label{fig:location}
\end{figure}
The detailed action tree for each action and dialogue type has been presented in the following subsections. Figure~\ref{fig:action_tree_ex} shows an example for a \textsc{build} action.

\begin{figure}[ht]
    \centering
    \small
    \begin{verbatim}
  0     1    2   3     4    5   6
"Make three oak wood houses to the
 7   8   9   10   11    12
left of the dark grey church."


{"dialogue_type" : "HUMAN_GIVE_COMMAND",
 "action_sequence" : [
   {
    "action_type" : "BUILD",
    "schematic": {
      "has_block_type": [0, [2, 3]],
      "has_name": [0, [4, 4]],
      "repeat": {
        "repeat_key": "FOR",
        "repeat_count": [1, 1]
      }},
     "location": {
       "relative_direction": "LEFT",
       "location_type": "REFERENCE_OBJECT",
       "reference_object": {
         "has_colour_": [0, [10, 11]],
         "has_name_": [0, [12, 12]] }
}}]}
    \end{verbatim}
    \vspace{-20pt}
    \caption{An example logical form. The spans are indexed as : [sentence\_number, [starting\_word\_index, ending\_word\_index]].  sentence\_number is 0 for the most recent sentence spoken in a dialogue and is 0 in our dataset since we support one-turn dialogues as of now.}
    \vspace{-8pt}
    \label{fig:action_tree_ex}
\end{figure}


\subsection{ Build Action}
This is the action to Build a schematic at an optional location. The Build logical form is shown in \ref{fig:build_dict} .

\begin{figure}[h]
    \centering
    \includegraphics[width=15cm,height=20cm,keepaspectratio]{figures/build.png}
    \caption{Details of logical form for Build}
    \label{fig:build_dict}
\end{figure}

\subsection{Copy Action}
This is the action to copy a block object to an optional location. The copy action is represented as a "Build" with an optional "reference object" . The logical form  is shown in \ref{fig:copy_dict}.

\begin{figure}[h]
    \centering
    \includegraphics[width=15cm,height=20cm,keepaspectratio]{figures/copy.png}
    \caption{Details of logical form for Copy}
    \label{fig:copy_dict}
\end{figure}

\subsection{ Spawn Action}
This action indicates that the specified object should be spawned in the environment. The logical form is shown in: \ref{fig:spawn_dict}

\begin{figure}[h]
    \centering
    \includegraphics[width=8cm,height=10cm,keepaspectratio]{figures/spawn.png}
    \caption{Details of logical form for Spawn action}
    \label{fig:spawn_dict}
\end{figure}


\subsection{ Fill Action}
This action states that a hole / negative shape at an optional location needs to be filled up. The logical form  is explained in : \ref{fig:fill_dict}

\begin{figure}[h]
    \centering
    \includegraphics[width=8cm,height=20cm,keepaspectratio]{figures/fill.png}
    \caption{Details of logical form  for Fill}
    \label{fig:fill_dict}
\end{figure}

\subsection{ Destroy Action}
This action indicates the intent to destroy a block object at an optional location. The logical form  is shown in: \ref{fig:destroy_dict}

Destroy action can have one of the following as the child:
\begin{itemize}
	\setlength\itemsep{0.0em}
	\item reference object
	\item nothing
\end{itemize}
\begin{figure}[h]
    \centering
    \includegraphics[width=8cm,height=20cm,keepaspectratio]{figures/destroy.png}
    \caption{Details of logical form  Destroy} 
    \label{fig:destroy_dict}
\end{figure}

\subsection{Move Action}
This action states that the agent should move to the specified location, the corresponding logical form  is in: \ref{fig:move_dict}

Move action can have one of the following as its child:
\begin{itemize}
	\setlength\itemsep{0.0em}
	\item location
	\item stop condition (stop moving when a condition is met)
	\item location and stop condition
	\item neither
\end{itemize}
\begin{figure}[h]
    \centering
    \includegraphics[width=15cm,height=20cm,keepaspectratio]{figures/move.png}
    \caption{Details of logical form  for Move action}
    \label{fig:move_dict}
\end{figure}

\subsection{ Dig Action}
This action represents the intent to dig a hole / negative shape of optional dimensions at an optional location. The logical form is in \ref{fig:dig_dict}

\begin{figure}[h]
    \centering
    \includegraphics[width=15cm,height=20cm,keepaspectratio]{figures/dig.png}
    \caption{Details of logical form for Dig action}
    \label{fig:dig_dict}
\end{figure}

\subsection{Dance Action}
This action represents that the agent performs a movement of a certain kind. Note that this action is different than a Move action in that the path or step-sequence here is more important than the destination. The logical form is shown in \ref{fig:dance_dict}

\begin{figure}[h]
    \centering
    \includegraphics[width=15cm,height=20cm,keepaspectratio]{figures/dance.png}
    \caption{Details of logical form for Dance action}
    \label{fig:dance_dict}
\end{figure}

\subsection{FreeBuild Action}
This action represents that the agent should complete an already existing half-finished block object, using its mental model. The logical form is explained in: \ref{fig:freebuild_dict}

FreeBuild action can have one of the following as its child:
\begin{itemize}
	\setlength\itemsep{0.0em}
	\item reference object only
	\item reference object and location
\end{itemize}

\begin{figure}[h]
    \centering
    \includegraphics[width=8cm,height=20cm,keepaspectratio]{figures/freebuild.png}
    \caption{Details of logical form for FreeBuild action}
    \label{fig:freebuild_dict}
\end{figure}

\subsection{ Undo Action}
This action states the intent to revert the specified action, if any. The logical form is in \ref{fig:undo_dict}.
Undo action can have on of the following as its child:
\begin{itemize}
	\setlength\itemsep{0.0em}
	\item target\_action\_type 
	\item nothing (meaning : undo the last action)
\end{itemize}
\begin{figure}[h]
    \centering
    \includegraphics[width=5cm,height=5cm,keepaspectratio]{figures/undo.png}
    \caption{Details of logical form for Undo action}
    \label{fig:undo_dict}
\end{figure}

\subsection{ Stop Action}
This action indicates stop and the logical form is shown in \ref{fig:stop_dict}
\begin{figure}[h]
    \centering
    \includegraphics[width=5cm,height=5cm,keepaspectratio]{figures/stop.png}
    \caption{Details of logical form for Stop action}
    \label{fig:stop_dict}
\end{figure}

\subsection{Resume Action}
This action indicates that the previous action should be resumed, the logical form is shown in: \ref{fig:resume_dict}
\begin{figure}[h]
    \centering
    \includegraphics[width=5cm,height=5cm,keepaspectratio]{figures/resume.png}
    \caption{Details of logical form for Resume action}
    \label{fig:resume_dict}
\end{figure}

\subsection{Get Memory Dialogue type}
This dialogue type represents the agent answering a question about the environment.
This is similar to the setup in Visual Question Answering. The logical form is represented in: \ref{fig:answer_dict}

Get Memory dialogue has the following as its children: filters, answer type and tag name.
This dialogue type represents the type of expected answer : counting, querying a specific attribute or querying everything ("what is the size of X" vs "what is X" )
\begin{figure}[h]
    \centering
    \includegraphics[width=12cm,height=12cm,keepaspectratio]{figures/answer.png}
    \caption{Details of logical form for Get Memory Dialogue}
    \label{fig:answer_dict}
\end{figure}


\subsection{Put Memory Dialogue}
This dialogue type represents that a reference object should be tagged with the given tag and the logical form is shown in: \ref{fig:tag_dict}

\begin{figure}[h]
    \centering
    \includegraphics[width=10.8cm,height=15cm,keepaspectratio]{figures/tag.png}
    \caption{Details of logical form for Put Memory Dialogue}
    \label{fig:tag_dict}
\end{figure}


\subsection{ Noop Dialogue}
This dialogue type indicates no operation should be performed, the logical form is shown in : \ref{fig:noop_dict}
\begin{figure}[h]
	\centering
    \includegraphics[width=5cm,height=5cm,keepaspectratio]{figures/noop.png}
    \caption{Details of logical form for Noop Dialogue}
    \label{fig:noop_dict}
\end{figure}

